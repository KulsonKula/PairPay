\documentclass[12pt,titlepage]{article}

\usepackage{geometry}
\geometry{
    a4paper,
    total={210mm,297mm},
    left=20mm,
    right=20mm,
    top=20mm,
    bottom=20mm,
}

	
\usepackage[]{subcaption}

\usepackage{pdflscape}

% Polski
\usepackage[]{polski} 
\usepackage[polish]{babel}

%do tabel
\usepackage{multirow}

% Pierwszy akapit - wcięty
\usepackage[]{indentfirst}

% Matematyka
\usepackage[]{amsfonts}

\usepackage[]{amsmath}

% Formatowanie
\usepackage{ragged2e}

% Tytuły sekcji
\usepackage{titlesec}
%\titleformat{\section}[block]{\Large\bfseries}{}{1em}{}

% <=
\usepackage{amssymb}

% eps
\usepackage{graphicx}
% \usepackage{subfigure}

% Tabele
\usepackage{array}

\usepackage[style=czech]{csquotes}

\renewcommand*{\thesubsubsection}{}

\usepackage{hyperref}
\hypersetup{
    colorlinks,
    citecolor=black,
    filecolor=black,
    linkcolor=black,
    urlcolor=black
}

\usepackage[numbered]{matlab-prettifier}
\lstset{
    literate={ą}{{\k{a}}}1
    {Ą}{{\k{A}}}1
    {ę}{{\k{e}}}1
    {Ę}{{\k{E}}}1
    {ó}{{\'o}}1
    {Ó}{{\'O}}1
    {ś}{{\'s}}1
    {Ś}{{\'S}}1
    {ł}{{\l{}}}1
    {Ł}{{\L{}}}1
    {ż}{{\.z}}1
    {Ż}{{\.Z}}1
    {ź}{{\'z}}1
    {Ź}{{\'Z}}1
    {ć}{{\'c}}1
    {Ć}{{\'C}}1
    {ń}{{\'n}}1
    {Ń}{{\'N}}1
}

\title{
\includegraphics[scale=0.75]{img/politechnika_sl_logo_bw_poziom_pl.eps}\\
\textbf{Wydział Automatyki, Elektroniki\\
i Informatyki}\\
\vspace*{1cm}
Technologie aplikacji internetowych \\ Projekt \\ PairPay

\vspace*{5cm}
}
\author{
Natalia Stręk,\\
Jakub Kula,\\
Paweł Wójtowicz
} 
\date{Gliwice 2024}

\begin{document}

\maketitle
\newpage
\section{Przedstawienie pomysłu na projektu}

Aplikacja do zarządzania wydatkami została zaprojektowana z myślą o osobach regularnie dzielących się kosztami, na przykład podczas wspólnych podróży, zakupów czy spotkań towarzyskich. Jej głównym celem jest uproszczenie procesu rejestrowania, monitorowania i rozliczania wydatków w grupach, co umożliwia efektywne zarządzanie finansami oraz minimalizuje ryzyko nieporozumień związanych z podziałem kosztów.\\

Użytkownicy mogą zarejestrować się i zalogować do systemu, uzyskując tym samym dostęp do chronionych funkcji aplikacji. Aplikacja pozwala na tworzenie nowych rachunków, do których można dodawać wydatki oraz uczestników, a następnie automatycznie dzieli koszty na podstawie wprowadzonych danych. Dzięki funkcji skanowania paragonów użytkownicy mogą łatwo importować informacje o wydatkach, eliminując konieczność ręcznego wprowadzania szczegółów.\\

Aplikacja oferuje także możliwość edytowania i usuwania rachunków oraz przeglądania historii wydatków, co zapewnia użytkownikom pełną kontrolę nad ich finansami. Dodatkowo, opcje filtrowania i sortowania rachunków, a także etykietowania wydatków, zwiększają użyteczność aplikacji, umożliwiając łatwe odnalezienie oraz zarządzanie danymi.\\

Wszystkie te funkcje sprawiają, że aplikacja jest nieocenionym narzędziem dla osób pragnących efektywnie zarządzać swoimi wydatkami, oszczędzając przy tym czas i minimalizując ryzyko błędów przy podziale kosztów.

\section{User Story}
\subsection{Para dzieląca się wydatkami}
Para, która mieszka razem, często robi wspólne zakupy. Zazwyczaj jedna osoba wybiera się na większe zakupy spożywcze, a druga później zwraca swoją część kosztów. Niestety, czasami pojawiają się problemy z ustaleniem, ile dokładnie każdy powinien zapłacić, bo paragonów przybywa, a nie zawsze jest czas, by od razu je podsumować. Dzięki aplikacji pariPay można na bieżąco dodawać wydatki do wspólnego rachunku i nie martwić się o szczegóły. 
\subsection{Rozliczenie rachunku w restauracji}
Grupa znajomych wybrała się na kolację do ulubionej restauracji. Po posiłku jedna osoba zdecydowała się opłacić cały rachunek. Zamiast przeliczać, kto co zamówił i ile powinien zapłacić, skorzystano z aplikacji pariPay. Wszystkie pozycje z paragonu zostały wprowadzone, przypisane do odpowiednich osób, a jednym kliknięciem wysłano znajomym wiadomość o rozliczeniu. Każdy może teraz bez problemu opłacić swoją część rachunku bez nieporozumień i przeliczania.
\subsection{Składka na imprezę z różnym rozliczaniem kosztów}
Podczas organizacji domowej imprezy grupa znajomych ustaliła, że złożą się na jedzenie i przekąski, a jedna osoba zadba o zakup napojów alkoholowych. Po imprezie pojawiła się potrzeba rozliczenia kosztów, jednak nie wszyscy spożywali alkohol, więc podział nie mógł być równy. W takim przypadku skorzystano z aplikacji pariPay, która pozwala na wprowadzenie wszystkich wydatków do wspólnego rachunku i precyzyjne przypisanie kosztów – np. pełne rozliczenie napojów alkoholowych tylko do tych, którzy je spożywali. Dzięki temu rachunek został podzielony sprawiedliwie, uwzględniając rzeczywiste wydatki każdej osoby.

\section{Specyfikacja wymagań aplikacji}
    \begin{table}[!h]
        \centering
        \resizebox{\textwidth}{!}{
            \begin{tabular}{|c|l|p{0.35\textwidth}|p{0.55\textwidth}|}
                \hline
                \textbf{Nr} & \textbf{Nazwa} & \textbf{Opis} & \textbf{Kryterium odbioru} \\ \hline
                \textbf{F\_1} & Logowanie i rejestracja & 
                Użytkownik może zakładać konto oraz logować się na nie. &
                System musi umożliwiać rejestrację, logowanie, oraz uzyskanie dostępu do chronionych funkcji aplikacji. Każda interakcja z chronionymi zasobami wymaga sprawdzenia tożsamości. Użytkownik musi także móc się wylogować, co kończy sesję i blokuje dostęp do zasobów. \\ \hline
                \textbf{F\_2} & Tworzenie nowego rachunku & 
                Użytkownik może tworzyć nowy rachunek, dodawać przedmioty oraz uczestników do podziału kosztów. &
                Użytkownik musi móc wprowadzić nazwę rachunku, dodać uczestników i szczegóły, a następnie zapisać rachunek. Rachunek musi być widoczny dla wszystkich dodanych uczestników. System musi poprawnie zapisywać dane rachunku oraz umożliwiać ich przeglądanie i edycję. \\ \hline
                \textbf{F\_3} & Dodanie nowych wydatków & 
                Aplikacja umożliwia dodawanie nowych wydatków do istniejących rachunków. &
                Użytkownik musi móc wprowadzić szczegóły wydatku, takie jak kwota, opis oraz przypisanie do uczestników. System musi aktualizować saldo rachunku i widok wydatków, a także potwierdzić dodanie wydatku. \\ \hline
                \textbf{F\_4} & Automatyczny podział kosztów & 
                Aplikacja automatycznie dzieli koszty między użytkowników. &
                System musi poprawnie obliczać i wyświetlać udział każdej osoby w rachunku, bazując na wydatkach oraz przypisanych uczestnikach. Użytkownik może wybrać metodę podziału (np. równy lub według udziałów) i otrzymać podsumowanie. System aktualizuje informacje po dodaniu nowych wydatków. \\ \hline
                \textbf{F\_5} & Automatyczne odczytywanie paragonów & 
                Aplikacja umożliwia wgranie zdjęcia paragonu, z którego odczytywane są informacje o kosztach. &
                System musi poprawnie rozpoznać i wydobyć kluczowe dane z paragonu (datę, kwotę, szczegóły wydatków). Użytkownik może przesłać obraz paragonu, a system zapisze te informacje w rachunku, potwierdzając ich dodanie. \\ \hline
                \textbf{F\_6} & Tworzenie grup znajomych & 
                Możliwość zapisywania grup osób, z którymi użytkownik często współdzieli rachunki. &
                Użytkownik może tworzyć grupy, nadając im nazwy i dodając członków z listy znajomych. System umożliwia edytowanie, usuwanie grup, oraz wyświetla wszystkie grupy użytkownika. \\ \hline
                \textbf{F\_7} & Edytowanie i usuwanie rachunku & 
                Aplikacja umożliwia edytowanie oraz usuwanie rachunków. &
                Użytkownik może edytować szczegóły rachunku (nazwa, uczestnicy, wydatki). System zapisuje zmiany i aktualizuje informacje, a także umożliwia usunięcie rachunku z odpowiednim potwierdzeniem i powiadomieniem uczestników. \\ \hline
                \textbf{F\_8} & Historia rachunków & 
                Aplikacja zapisuje poprzednie rachunki, umożliwiając przeglądanie historii. &
                Użytkownik ma możliwość wyświetlania listy wcześniejszych rachunków, wraz z informacjami o nazwie, dacie oraz uczestnikach, a także przeglądania szczegółów każdego rachunku. \\ \hline
                \textbf{F\_9} & Filtrowanie i sortowanie rachunków & 
                Aplikacja umożliwia filtrowanie i sortowanie rachunków według różnych kryteriów. &
                Użytkownik może stosować różne filtry, takie jak data, uczestnicy, kwota. System aktualizuje widok rachunków zgodnie z wybranymi kryteriami, umożliwiając sortowanie według wybranych parametrów. \\ \hline
                \textbf{F\_10} & Etykietowanie rachunków & 
                Aplikacja umożliwia przypisanie etykiet do wydatków, określających ich typ (np. rozrywka, transport, dom). &
                System musi umożliwiać łatwe przypisywanie i przeglądanie etykiet, co ułatwia klasyfikację wydatków w rachunkach. \\ \hline
            \end{tabular}
        }
        \caption{Opis funkcjonalności aplikacji oraz kryteriów odbioru}
        \label{tab:funkcjonalnosci}
    \end{table}
    
    \begin{table}[!h]
        \centering
        \renewcommand{\arraystretch}{1.3} 
        \resizebox{\textwidth}{!}{
            \begin{tabular}{|c|l|p{0.7\textwidth}|}
                \hline
                \textbf{Kod}    & \textbf{Nazwa}       & \textbf{Opis} \\ \hline
                \textbf{NF\_1}  & Wydajność            & Aplikacja powinna ładować strony i przeprowadzać operacje w sposób płynny, zapewniając komfortowe doświadczenie użytkownika. \\ \hline
                \textbf{NF\_2}  & Bezpieczeństwo       & System musi zapewniać szyfrowanie danych użytkowników zarówno podczas przesyłania, jak i w spoczynku, chroniąc je przed nieautoryzowanym dostępem. \\ \hline
                \textbf{NF\_3}  & Użyteczność          & Interfejs użytkownika powinien być prosty i intuicyjny, zawierając jasne instrukcje dotyczące dodawania rachunków, etykietowania i podziału kosztów, aby zminimalizować krzywą uczenia się dla nowych użytkowników. \\ \hline
                \textbf{NF\_4}  & Dokumentacja         & Aplikacja powinna posiadać szczegółową dokumentację zarówno dla użytkowników, jak i administratorów, opisującą wszystkie funkcje oraz procedury użytkowania. \\ \hline
            \end{tabular}
        }
        \caption{Opis wymagań niefunkcjonalnych aplikacji}
        \label{tab:non-functional-requirements}
    \end{table}
        
    


\newpage
\section{Skala MoSCow}
    
\begin{table}[ht]
    \centering
    \renewcommand{\arraystretch}{1.3} 
    \begin{tabular}{|c|c|}
        \hline
        \textbf{Funkcjonalność} & \textbf{Skala MoSCoW} \\ \hline
        \textbf{F\_1}           & Must                  \\ \hline
        \textbf{F\_2}           & Must                  \\ \hline
        \textbf{F\_3}           & Must                  \\ \hline
        \textbf{F\_4}           & Should                \\ \hline
        \textbf{F\_5}           & Could                 \\ \hline
        \textbf{F\_6}           & Should                \\ \hline
        \textbf{F\_7}           & Should                \\ \hline
        \textbf{F\_8}           & Should                \\ \hline
        \textbf{F\_9}           & Could                 \\ \hline
        \textbf{F\_10}          & Could                 \\ \hline
        \textbf{NF\_1}          & Must                  \\ \hline
        \textbf{NF\_2}          & Must                  \\ \hline
        \textbf{NF\_3}          & Should                \\ \hline
        \textbf{NF\_4}          & Could                 \\ \hline
    \end{tabular}
    \caption{Klasyfikacja funkcjonalności według skali MoSCoW}
    \label{tab:moscow-classification}
\end{table}

\section{Diagram przypadków użycia}
\includegraphics[width=1\textwidth]{UserDiagram.png}

    \end{document}