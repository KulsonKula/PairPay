\documentclass[12pt,titlepage]{article}

\usepackage{geometry}
\geometry{
    a4paper,
    total={210mm,297mm},
    left=20mm,
    right=20mm,
    top=20mm,
    bottom=20mm,
}

	
\usepackage[]{subcaption}

\usepackage{pdflscape}

% Polski
\usepackage[]{polski} 
\usepackage[polish]{babel}

%do tabel
\usepackage{multirow}

% Pierwszy akapit - wcięty
\usepackage[]{indentfirst}

% Matematyka
\usepackage[]{amsfonts}

\usepackage[]{amsmath}

% Formatowanie
\usepackage{ragged2e}

% Tytuły sekcji
\usepackage{titlesec}
%\titleformat{\section}[block]{\Large\bfseries}{}{1em}{}

% <=
\usepackage{amssymb}

% eps
\usepackage{graphicx}
% \usepackage{subfigure}

% Tabele
\usepackage{array}

\usepackage[style=czech]{csquotes}

\renewcommand*{\thesubsubsection}{}

\usepackage{hyperref}
\hypersetup{
    colorlinks,
    citecolor=black,
    filecolor=black,
    linkcolor=black,
    urlcolor=black
}

\usepackage[numbered]{matlab-prettifier}
\lstset{
    literate={ą}{{\k{a}}}1
    {Ą}{{\k{A}}}1
    {ę}{{\k{e}}}1
    {Ę}{{\k{E}}}1
    {ó}{{\'o}}1
    {Ó}{{\'O}}1
    {ś}{{\'s}}1
    {Ś}{{\'S}}1
    {ł}{{\l{}}}1
    {Ł}{{\L{}}}1
    {ż}{{\.z}}1
    {Ż}{{\.Z}}1
    {ź}{{\'z}}1
    {Ź}{{\'Z}}1
    {ć}{{\'c}}1
    {Ć}{{\'C}}1
    {ń}{{\'n}}1
    {Ń}{{\'N}}1
}

\title{
\includegraphics[scale=0.75]{img/politechnika_sl_logo_bw_poziom_pl.eps}\\
\textbf{Wydział Automatyki, Elektroniki\\
i Informatyki}\\
\vspace*{1cm}
Technologie aplikacji internetowych \\ Projekt \\ PairPay

\vspace*{5cm}
}
\author{
Natalia Stręk,\\
Jakub Kula,\\
Paweł Wójtowicz
} 
\date{Gliwice 2024}

\begin{document}

\maketitle
\newpage
\section{Architektura}
Przeanalizuj, jak zrealizować wymagania od strony technicznej. Jakie komponenty składają się na system? Czy będzie to jeden serwer czy kilka? Jakie technologie będą używane?
\begin{enumerate}
    \item Architektóra serwera: mikroserwisy -
    \item Architektóra API: RESTful - 
    \item Modele skalowaności - kubernetes ???????????????????????
    \item Zarządzanie stanem - ???????????????????????????
    \item Chmura i infrastryktura - ??????????????
\end{enumerate}

\subsection{Diagram komponentów}
\subsection{Diagram sekwencji}
\section{Użyte technologie w projekcie}
\begin{enumerate}
    \item Python, Flask - Flask jest jednym z czterech największych frameworków backendowych w Pythonie, wyróżniającym się szybkością wdrażania oraz skalowalnością. Jednym z istotnych atutów Flask jest jego rozbudowana i aktywna społeczność, co sprawia, że w razie potrzeby łatwo jest znaleźć pomoc i zasoby wspierające rozwój projektu. Dzięki temu framework ten idealnie nadaje się do tworzenia małych i średnich projektów, umożliwiając szybkie prototypowanie i iterację. Istotnym elementem jest także to, że Flask bazuje na Pythonie – języku programowania znanym ze swojej czytelności oraz obszernego ekosystemu bibliotek i narzędzi. Python, będący jednym z najłatwiejszych do nauki języków, oferuje jednocześnie szerokie możliwości przetwarzania danych, co stanowi istotną zaletę w kontekście projektów wymagających analiz i przetwarzania informacji.
    \item BAZA danych
    \item FRONT END
\end{enumerate}
\section{Bezpiczeństwo}
\subsection{Zabezpieczenia logowania i autoryzacji}
\subsection{Zarządzanie sesjami}
\subsection{Zabezpieczenia bazy danych}
\subsection{Zarządzanie dostępem}

    \end{document}