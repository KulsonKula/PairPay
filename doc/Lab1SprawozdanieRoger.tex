%! Author = olive
%! Date = 29/10/2023

% Preamble
% Preamble
\documentclass[12pt,titlepage]{article}

\usepackage{geometry}
\geometry{
    a4paper,
    total={210mm,297mm},
    left=20mm,
    right=20mm,
    top=20mm,
    bottom=20mm,
}
% % Polski
\usepackage[]{polski}
\usepackage[polish]{babel}

\usepackage{graphicx}
\usepackage{hyperref}
\usepackage{epstopdf}
\usepackage{amsmath}
\usepackage{subfig}
\usepackage[utf8]{inputenc}
\usepackage[T1]{fontenc}
\usepackage{standalone}
\usepackage{tikz}
\usepackage{tabularx}
\usepackage{multirow}
\usepackage{textcomp}
\usepackage{pgfplots}
\usepackage{listings}
\usepackage{lscape}
\usepackage{pdfpages}

\usepackage[]{subcaption}

\usepackage{pdflscape}

% Pierwszy akapit - wcięty
\usepackage[]{indentfirst}

% Matematyka
\usepackage[]{amsfonts}

% Formatowanie
\usepackage{ragged2e}

% Tytuły sekcji
\usepackage{titlesec}

\usepackage{array}

\usepackage[style=czech]{csquotes}





% Document

\title{
    \includegraphics[scale=0.75]{img/politechnika_sl_logo_bw_poziom_pl-eps-converted-to}\\
    \textbf{Wydział Automatyki, Elektroniki\\
    i Informatyki}\\
    \vspace*{1cm}
    Metody sztucznej inteligencji \\ Laboratorium 1 \\ Sieci neuronowe jednokierunkowe

    \vspace*{5cm}
}
\author{
    Jakub Kula,
    Oliver Davis
}
\date{Gliwice 2023}

\begin{document}
    \maketitle

    \tableofcontents
    \newpage
        \section{Wpływ postaci funkcji aktywacji}

        \subsection{Wykres funkcji pierwotnej}

    Badanymi funkcjami są dwie podane funkcje poniżej.

        Funkcja pierwotna jest opisana następującym wzorem:
        \begin{align*}
            y &= \sin(0.9x) + \cos(2x) \\
        \end{align*}

        Która również posiada swoją wersję zaszumioną:
        \begin{align*}
            Y &= \sin(0.9 \cdot X) + \cos(2 \cdot X) + 0.1 \cdot \text{randn}(\text{size}(X)) \\
        \end{align*}

        gdzie:
        \begin{align*}
            x &= \left(-\frac{\pi}{2}\right) : \frac{\pi}{100} : \left(2.5 \pi\right)
        \end{align*}

    Implementacja tych funkcji w środowisku matlab jest realizowana poprzez następujący fragment kodu:\newline
    \lstinputlisting[language=Matlab, firstline=258, lastline=264]{../kod.m}

    \newpage
    \begin{landscape}
        \includegraphics[width=\linewidth]{img/pierowotna.png}

        \subsection{Purelin}
        \subsubsection{Porównanie ze zbiorem uczącym}
        \includegraphics[width=\linewidth]{img/purelin_uczacy.png}\\
        \subsubsection{Porównanie ze zbiorem testujacym}
        \includegraphics[width=\linewidth]{img/purelin_testujący.png}\\
        \subsubsection{Porównanie wyników symulacji na podstawie zbioru uczacego i zbioru tesującego}
        \includegraphics[width=\linewidth]{img/purelin_porownanie.png}\\

        \subsection{Tansig}
        \subsubsection{Porównanie ze zbiorem uczącym}
        \includegraphics[width=\linewidth]{img/tansig_uczacy.png}\\
        \subsubsection{Porównanie ze zbiorem testujacym}
        \includegraphics[width=\linewidth]{img/tansig_testujący.png}\\
        \subsubsection{Porównanie wyników symulacji na podstawie zbioru uczacego i zbioru tesującego}
        \includegraphics[width=\linewidth]{img/tansig_porownanie.png}\\

        \subsection{Logsig}

        \subsubsection{Porównanie ze zbiorem uczącym}
        \includegraphics[width=\linewidth]{img/logsig_uczacy.png}\\
        \subsubsection{Porównanie ze zbiorem testujacym}

        \includegraphics[width=\linewidth]{img/logsig_testujący.png}\\

        \subsubsection{Porównanie wyników symulacji na podstawie zbioru uczacego i zbioru tesującego}
        \includegraphics[width=\linewidth]{img/logsig_porownanie.png}\\

        \subsection{Porówniane wyników dla kazdej z metod dla 100 neutonów}
        \includegraphics[width=\linewidth]{img/porownanie_3_metod.png}\\


        \section{Zależności Średniego błedu MSE w zależności od liczby neuronów}

        \subsection{Purelin}
        \includegraphics[width=\linewidth]{img/MSE_purelin.png}\\

        \subsection{Tansig}
        \includegraphics[width=\linewidth]{img/MSE_tansig.png}\\

        \subsection{Logsig}
        \includegraphics[width=\linewidth]{img/MSE_logsig.png}\\
    \end{landscape}

    \section{Analiza wyników i wnioski}
    Analizując wykresy, odrazu jesteśmy w stanie stwierdzić ze funkcja aktywacji "Purelin" najgorzej poradziła sobie z zadaniem aproksymacji funkcji nieliniowej.\newline

    Niezależnie od ilości neuronów w sieci, średnia wartość wskaźnika MSE jest taka sama, \\a kształt naszej funkcji po symulacji nie oddaje charakteru funkcji pierwotnej.\newline

    Funckje "Tansig" oraz "Logsig" poradziły sobie z zadaniem znacząco lepiej.\newline

    Można zauważyć znaczny spadek wskaźnika MSE wraz ze wzrostem liczby neuronów w sieci, który osiąga bardzo małe wartości w okolicach 0.01 już przy 9 neuronach w sieci.\newline

    Dla wszystkich trzech funkcji aktywacji mozna zauważyć też, ze wartość wskaźnika MSE dla zbioru tesotwego jest większa niż dla zbioru uczącego. \newline

    Warto zwrócić uwagę także na kształt funkcji po symulacji, która wraz ze wzrostem liczby neuronów, wykres zaczał robić sie bardzej ostry, a dla mniejszej ilości, nabiera kształt bardzej gładki.\newline


\end{document}